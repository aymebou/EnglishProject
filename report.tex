
\documentclass[a4paper, 11pt]{article} % Font size (can be 10pt, 11pt or 12pt) and paper size (remove a4paper for US letter paper)

\usepackage{hyperref}
\usepackage{caption}
\usepackage[protrusion=true,expansion=true]{microtype} % Better typography
\usepackage{graphicx} % Required for including pictures
\usepackage{wrapfig} % Allows in-line images

\usepackage{mathpazo} % Use the Palatino font
\usepackage[T1]{fontenc} % Required for accented characters
\usepackage{multicol}


\usepackage{graphicx}
\usepackage{subcaption}
\usepackage{tikz}

\linespread{1.05} % Change line spacing here, Palatino benefits from a slight increase by default

\makeatletter
\renewcommand\@biblabel[1]{\textbf{#1.}} % Change the square brackets for each bibliography item from '[1]' to '1.'
\renewcommand{\@listI}{\itemsep=0pt} % Reduce the space between items in the itemize and enumerate environments and the bibliography

\renewcommand{\maketitle}{ % Customize the title - do not edit title and author name here, see the TITLE block below
\begin{flushright} % Right align
{\LARGE\@title} % Increase the font size of the title

\vspace{30pt} % Some vertical space between the title and author name

{\large\@author} % Author name
\\\@date % Date

\vspace{20pt} % Some vertical space between the author block and abstract
\end{flushright}
}

%----------------------------------------------------------------------------------------
%	TITLE
%----------------------------------------------------------------------------------------

\title{\textbf{The life of a star}\\ % Title
Autonomous project : English} % Subtitle

\author{\textsc{Pauline Dubuc, Suzon Thenaux, Aymeric Bouzigues} % Author
\\{\textit{Mines Nancy}}} % Institution

\date{\today} % Date

%----------------------------------------------------------------------------------------

\begin{document}

\maketitle % Print the title section

%----------------------------------------------------------------------------------------
%	ABSTRACT AND KEYWORDS
%----------------------------------------------------------------------------------------

%\renewcommand{\abstractname}{Summary} % Uncomment to change the name of the abstract to something else

\begin{abstract}

This paper aims to give to the reader a clear understanding of what are the different types of star, and why it is important to pu them in different boxes. It has a scientific point of view, but requires no mthematical background.

\end{abstract}


\begin{figure}[h]
\centering
\includegraphics[height=10cm]{star-intro}
\captionsetup{labelformat=empty}
\caption{Star cluster captured by the Hubble Telescope}
\end{figure}
\newpage
\tableofcontents
\newpage
\section*{Introduction}

\subsection*{History of astronomy and astrophysics}

Astronomy is one of the oldest science. People have always studied and observed the sky and planets. Its first mean is to understand the origins and the evolution of the Universe, it also allows to study the movements, positions and behaviors of celestial objects. Concerning astrophysic, it's a more modern science. Astrophysic studies the physical aspect of cosmic phenomenons such as birth and evolution of stars and galaxies. In this project, we aimed to summary the astrophysic of stars : from the birth to the destruction.
\paragraph*{}
In the antiquity, men observed the Sun's position through the days, and star's positions during the night to measure time. Those luminaries also revealed themselves to be useful for navigation. It is thought that Babylonians started studying the sky in 3000 BC. 
\paragraph*{}
Mathematics theories were developed by ancient civilisations ( greeks, indians and chinese). They tempted to predict solar and lunar eclipses. That is how Aristote discovered that Earth was round, Erastothenes calculated the Earth diameter etc.
The first telescope was created by Isaac Newton in 1607. Before this, Galileo used a refractor bezel. In the XVIII century, the British William Herschel inventoried 848 stars. In 1781, he discovered Uranus. At this time, only 7 planets have been observed. Then in the XIX century, telescopes with prisms allowed scientifics to study chemical composition of stars.


\subsection*{The importance of stars in the universe}

Stars are everywhere in the universe. They are far from being only the little dots we can see at night when the sky is clear. They are the power plants of the universe and its most common inhabitant. They also are the forge of new matter, as every atom in your body that isn't hydrogen was once created in the heart of a star and expelled in the universe when the star exploded, to later coalesce in matter cluster, forming planets and maybe later, you.
These huge spheres of burning gas are essential the universe, and is essential for us, as the sun is our biggest energy source and our only light source. 

\paragraph*{}
With $10^{22}$ stars in the universe\footnote{About the same as the number of sand grains in planet earth.}, this makes \textit{the} most frequent celestial body in space, some of them living as much as 100 billion years while others explode being only few million years old, their complexity is yet to be fully understood.



\section{The theory principles behind a star}

\subsection{The creation process}
Stars are created every day, by a random process. They don't arise ex-nihilo, but are formed in \textbf{nebulae} that can be as much as 300 light-years\footnote{ 1 light year is over 9.5 Billion kilometers, in comparison, earth-sun is 0.15 Billion kilometers.} wide. In a nebula, there are huge clouds of gas in movement. One slight event can provoke the creation of a star, while nothing can happen in millions of years, making the creation process very random.

\paragraph*{}
In the stars, gas clouds and the universe in general, everything  a question of balance. The clouds within the nebula must maintain balance between their inner pressure and their temperature. These are explained by the \textbf{equation of state of an ideal gas} : 
\label{GP}
\begin{multicols}{2}

\begin{equation}
P\cdot V = n \cdot R \cdot T
\end{equation}
\vspace{2cm}
\begin{itemize}
\item P : the pressure
\item V : the volume
\item n and R : constants of the gas cloud.
\item T : the temperature 
\end{itemize}
\end{multicols}

The temperature within the nebula tends to increase, due to light rays from nearby stars providing a slow heat source, just like the sun with earth. As we can see in this equation, an increase in temperature will trigger an increase in volume and pressure. As the pressure steadily increases along with the temperature, the particles within the gas speed up, as seen in the \textbf{equation of particle speed in function of temperature} :

\begin{multicols}{2}

\begin{equation}
v = C \cdot \sqrt{T}
\end{equation}
\vspace{2cm}
\begin{itemize}
\item v : the particle speed
\item T : the temperature
\item C : constant of the gas cloud. 
\end{itemize}
\end{multicols}

As the particle speeds up, more and more collision occur within the cloud and just like a lighter, the cloud start to ignite and the star is born. The \textit{ignite} is actually the fusion\footnote{Explained in next section} process beginning which is the essence of what a star is and why it \textit{lives}.

\subsection{The fusion process}
The fusion process is the essence of the star, it makes it shine, and allows it not to collapse into a \textbf{black hole}\footnote{A black hole is a object with huge density, when gravity collapses atoms on themselves, they are explain in section REF}. Atoms, and matter in general contain energy, energy and mass are essentially the same thing\footnote{Einstein exposed a link between them : $E = mc^2$ chich translates that you can actually convert energy and mass easily.}. 


\begin{figure}[h]
\centering
\includegraphics[height = 6cm]{binding-energy}
\caption{Energy contained in atoms}
\end{figure}

As said earlier, stars are mainly composed of hydrogen, and they use it to gather energy, the energy they need to shine. As shocks occur, the \textbf{Hydrogen in the star becomes Helium}. As we can see in Figure 2, Helium contains less energy than hydrogen, which means that the energy difference when the transformation occurs is released in light and heat.

The fusion process follows this pattern :
\begin{figure}[h]
\centering
\includegraphics[height = 6cm]{fusion}
\caption{Hydrogen fusion}
\end{figure}

The interesting fact to notice is that fusion auto-fuels itself : it needs heat and the process generates heat.
\paragraph*{}
The star's boring and repetitive life is to maintain an equilibrium between the fusion process and the gravity, the first one expanding the atoms and the star, and the second one collapsing it on itself.

\begin{figure}[h]
\centering
\includegraphics[height = 4cm]{balance}
\caption{Balance between fusion and gravity}
\end{figure}

Since in the world, there are a lot of different types of star, we will first begin by explaining the different classifications and how they are important.

\subsection{The solar flares : how this phenomenon was discovered}

The balance between the star's gravity and its pressure was discoverd long ago by Galileo thanks to solar flares.

Solar flares happens in the atmosphere of the sun. It is a brutal release of energy at the surface of the sun : more specifically, it is an emission of light and particles. For the biggest flares, plasma\footnote{The plasma here isn't what is in your TV, or in your blood, but rather a very hot state of matter, just like solid or liquid, but where particles are so hot they form a sort of soup. There can be massive amounts of matter rejected in outer space, as much as 10 million tons. This seems huge but it is nothing compared to the actual size of the star.} can be rejected too.. This flow of charged particles can have huge consequences on our systems and scientist are looking forward to forecast them so that they study carefully and meticulously this phenomenon. An example of the consequences of solar flares is perturbation of a satellite or the provocation of problems on electrical networks and phone networks.

\begin{figure*}[h!]
    \centering
  \begin{subfigure}[b]{0.45\textwidth}
      \includegraphics[height=3.7cm]{solar}
      \caption{Solar flares phenomenon}
  \end{subfigure}%
  ~ 
  \begin{subfigure}[b]{0.45\textwidth}
      \centering
      \includegraphics[height=3.7cm]{solar-magn}
      \caption{Solar flare's manetic fields}
  \end{subfigure}
\end{figure*}



\paragraph*{}
Indeed, magnetic fields created during a solar flare are turbulent and unpredictable. The process is that field lines wind and tangle like a rope (as shown in the following picture). They ultimately break and the energy contained in the structure is released when accelerated charged particles meet the gas. The flare do not occur at the very surface of the sun but a little further, in a layer called the corona of the sun. 

This scheme points out the winded magnetic fields which create energetical flows. When it breaks, the plasma is ejected with its energy. This is one of te reason matter is thrown out of the star during its lifetime.\footnote{The second one being that they emit light, and that light and matter are essentially the same thing.} 

Even if the scientist are sure of the flare's causes, they are still working on the mechanisms behind it, for example they still do not know how the magnetic field increase the particle's speed. 

\paragraph*{}
Another consequence of the solar flares are the beautiful aurorapolar caused by the solar winds. High speed very magnetic particles are forced in outer space in random directions. If some of them reach earth, they are directed towards the poles thanks to earth's magnetic field's shape. The remaining particles then hit the pole's atmosphere at full speed, an emit light of unusual colors as they are unusual light particles.

\newpage
\begin{figure}[h]
\centering
\includegraphics[height = 4cm]{aurora}
\caption{The earth's magnetic field and magnetic light particles trying to flow through it}
\end{figure}

\subsection{The fusion process in the hands of man}

In short words, the fusion process basically is mixing up small atoms to make bigger, more stable atoms. This releases energy in the air, which is either in heat of in light form. You might have heard more about fission than fusion, the process used in nuclear power plants to give us energy and electricity. Fission is the exact opposite of fusion. The principle is to take heavy atoms and split them into two smaller atoms. The problem with this technique is that the resulting atoms are always highly unstable, so unstable that they threaten to break at any time. If one of these atoms break near you, they will eject neutrons, X rays and Gamma rays. Gamma rays are so thin that they go through every one of you cell without any problem, but they can hit the smallest, dearest thing within your cells : your \textbf{DNA}. This is the reason radioactive waste are so dangerous and no one can get near them without a radioactive suit.
\paragraph*{}
Though fusion and fission work the same way, the waste produced by fusion in iron, which is \textit{the} most stable element in the universe. Moreover, the fuel for fusion is hydrogen, the most common atom in the universe\footnote{For fission, we need Uranium or Thorium, which is very rare.}. The icing on the cake is that fusion produces way more energy than fission. 
\paragraph*{}
Then, why are we not using fusion in nuclear powerplants ? We are, but currently it is just a project, named project \textbf{ITER}\footnote{The ITER project was born in France, https://www.iter.org/}, designed by a team of researcher, and it is not yet advantageous to use it as it consumes more energy than produced. The main problem with fusion is that we are afraid of it. Is produces too much energy, and it is self-fueld, which means once started in a medium, is is not stopable until the fuel, hydrogen, is gone. The fear is that we have to heat the medium to 3 000 degrees before it is ignited and fusion starts. There is literally nothing that can contain these kind of temperature and it could melt a whole building, or mountain. It is like trying to cover the sun with a blanket hoping it will not catch fire. Nevertheless, fusion is clearly the future of energy, since it is cleaner and more efficient. To make you understand how efficient it is, for 1kg of hydrogen transformed into hydrogen, the energy released is \textbf{more than 3 times the energy consumption of Paris in 2016}. This is the reason people are afraid of fusion, it is too efficient and produces so much energy, it is hard to use it all. But imagine, for a spaceship, or a plane, having 1 gram of hydrogen would be sufficient for a hole trip.

\section{Introduction to stellar classifications}



Stars can be sorted out according to a variety of different physical parameters:

\begin{itemize}
\item Their position in the \textbf{lifecycle of the star} (e.g main sequence or dying).
\item Their color (refered as \textbf{spectral type} later in this document.
\item \textbf{The prevalence}.
\item \textbf{The temperature}. 
\item \textbf{The luminosity}.
\item \textbf{The radius}.
\item \textbf{The mass}.
\item \textbf{The age}.
\end{itemize}

The most used system is the M-K\footnote{Named after their inventors, Morgan and Keenan} system, which is a lettering system that corresponds to the star's spectrum\footnote{Color} and ergo its temperature. But the color only is not enough to put the stars in boxes and give them labels.  

\subsection{Color and temperture}

The colors and temperature of a star are intrinsicaly connected. Scientist use spectrum analysis to determine a star's temperature and can also get an approximation of its mass, radius and luminosity, making it the most important criteria in the MK classification.



\begin{figure}[h]
\centering
\includegraphics[height = 4cm]{M-O}
\caption{Star lettering system}
\end{figure}

The coolest stars are M-star and O are the hottest. Our system's beloved star, the sun is a G type of star.


\subsection{Radius and mass}

As very often in astrophysics, the units are specific. Since we will give the radius of the different stars using these specific units to fit the enormous values. The most used unit for interstellar distances is the lightyear, however, it doesn't fit the star requirement, a light year is the distance traveld by a light photon in a hole year\footnote{Remember that light travels at 300 000 km per second.} whereas it only takes 7 minutes for light to reach earth from the sun. The usual standard for creating units is taking historical references (like farenheit for an example). Ergo, for stars, the reference is the sun. A radius of 1 means that is has the same size as the sun. same goes for the mass, the reference value of one corresponds to the sun's mass.

The masses and radius can go very high and differ a lot from one star to another. To get a grasp of those differences, on this rudimentary 3D model, there are different celestial bodies, on every box, the far right body is reported on the next box's far left side to understant the scale difference.

\newpage

\begin{figure}[h]
\centering
\includegraphics[width = 9cm]{scale}
\caption{Table of different scales}
\end{figure}



\subsection{Luminosity}

Curious as it may sound, the luminosity and the color aren't exactly corresponding, and it is needed to have both to correctly describe a star. The luminosity actually takes into account the magnetic field of the star making it even more important. As in the previous section, the only change is in the units, however, not only a linear change here is needed to correspond to the \textit{usual} metric system.
\paragraph*{}
The conversion to obtain the luminosity is given here :
\begin{multicols}{2}

\begin{equation}
\frac{L_\star}{L_\odot} = 10^{0.4\left(M_\mathrm{bol,\odot} - M_\mathrm{bol,\star}\right)}
\end{equation}
\vspace{2cm}
\begin{itemize}
\item $L_\star$ : the star's luminosity
\item $L_\odot$ : the sun's luminosity
\item $M_\mathrm{bol,\odot}$ : magnitude of the star.
\item $M_\mathrm{bol,\star}$ : magnitude of the earth
\end{itemize}
\end{multicols}

\subsection{The prevalence}

The prevalence is the proportion of this specific star amongst every star in the universe. It is usually given in a percentage and a higher percentage indicates that this star is more comon.	
\newpage
\subsection{Corresponding table of different criteria in an MK table}

\begin{figure}[h]
\centering
\includegraphics[width = \textwidth]{criteria}
\caption{Different criteria in MK category system}
\end{figure}

\subsection{Classification in star population}

A group of stars sharing the same properties is called a stellar population. 
Differentiating them is quite difficult due to the number of stars and their close specification. Putting them in closed boxes seems like an impossible task. Stellar populations are so similar that specialists talks of a \textit{``continuum characteristics that reflect the changes in star formation with time''}. Each stellar population is a witness of an event occured previsously in the Universe's life.




\paragraph*{}
Since stars are the most common object in space, the structure of star populations define the structure and shape of the galaxies. There are three main different shape for a star population :

\begin{itemize}
 \item The disk population : rotating and flattened region
\item The bulge population : rounded and central region
\item The halo population : far outer regions (ellipsoidal geometry)

\end{itemize}

A galaxy can have a form that is a mix of all those shapes, or any individual one.
\newpage
\begin{figure}[h]
\centering
\includegraphics[height = 6cm]{shapes}
\caption{Different shapes for a galaxy and a star population}
\end{figure}

An example of a galaxy that is a great mix of all those shapes is the \textbf{Sombrero} galaxy. 

Disk and bulge populations are often rich in heavy elements (above helium on the periodic table). Halo populations seem to be very poor in heavy elements.


\section{The end of the equilibrium and the end of the star}

Unfortunately for the star, this balance isn't forever and decays with time. In the beginning, the star fuses Hydrogen, then Helium, then all the way to Iron, which is \textbf{the most stable element in the universe}, the star cannot get anymore energy because it has reached the end of the fusion process.
\begin{equation}
H \rightarrow He \rightarrow C\ (Carbon) \rightarrow Ne\ (Neon)\rightarrow O\ (Oxygen)\rightarrow Fe\ (Iron)
\end{equation}

Then, the fusion process ends, and the only force applying on the star is its own gravity. The star then collapses on itself and begins a long journey to delay a unavoidable end for the star. It can take different path, that vary with the star's mass. For the fate of the star, only the inital mass has a noticeable influence. In this paper, the stars will be explained starting by lighter stars all the way to the heaviest, producing a \textbf{black hole}.

The most common, for smaller stars is the \textbf{brown dwarf}. This ``star'' isn't quite other stars as it sits inbetween a star and a planet, but classifications usually take these into account, as we will. The fate of starts as described in the first section \textit{per say} will be described in the section \ref{whiteDwarf}.

\newpage

\subsection{Brown dwarf}

\begin{figure}[h]
\centering
\includegraphics[width = 9cm]{brown-sheet}
\caption{The place of the brown dwarf in the classification }
\end{figure}

A brown dwarf is a failed star, heavy enough to have ignited deuterium fusion, but couldn't manage to fuse hydrogen and fails shining like other stars. They weigh less than 0.07 times the mass of the sun or around 75 times Jupiter's mass.


The Brown Dwarf were discovered in 1995 for the first time: they are difficult to find because they are so \textit{dark} that an infrared detector is needed to detect them. 
The main reason why scientist consider these objects as stars is that they are born in the same way: the gravitational collapse of a cloud of gas. Because of their mass they rapidly stop emitting light and emit instead infrared radiation. The temperature of the Brown Dwarf are between 400K and 2500K. 


The reason why scientist were extremely interested in these Brown Dwarf is that they thought that Brown Dwarf were the \textbf{hidden matter of the universe}. Recent studies found that there is invisible and unknown matter in the universe\footnote{ This matter is now know as dark matter and makes up around 27\% of the mass of the universe. It is the reason why the galaxies, planets, stars and globally the universe sticks together even if all the stellar object spins with high speed.}. One of the hypothesis when Brown Dwarf were discovered is that numerous Brown Dwarf were the dark matter of the universe because they were so hard to detect. this hypothesis turned out wrong because not enough brown stars exist in the universe. To this day, the origin of dark matter remains unknown.

\subsection{Red dwarf}
\newpage
\begin{figure}[h]
\centering
\includegraphics[width = 12cm]{sheet-RD}
\caption{The place of the red dwarf in the classification }
\end{figure}

The red dwarf are the most common stars of the universe. They are said to represent around 80\% of the universe. They spectral type is K or M and their mass is a little lower than the Yellow Darf’s : between 0.08 and 0.8 mass of the sun. Because of their low mass, their temperature is very low: around 3500K. They are big and hot enough to produce helium from hydrogen but the fusion is much slower than for other stars like Yellow Dwarf. Their lifetime is therefore very long : between 10 billion years and  and a thousand billion years. 
\paragraph*{}
They were the first star detected that wasn't within our solar system, thanks to their high density in the universe, and their suitable for detection size. This is the star scientist know th best and were able study the most. These red dwarves are the textbook stars, they go through fusion but they are not massive enough to start fusing carbon and the process ends. As the process ends, the gravity pushes inwards the star, and it collapses on itself. Either the star becomes a white dwarf\footnote{See section \ref{whiteDwarf}} or it bounces back and it becomes a red giant. 
\paragraph*{}
Even though red dwarves are easy to observe, it took 50 years after their discovery todiscover that their lifetime was not older than the universe. Since a red dwarf lives for 10 billion years at least, the universe being 17 billion old, a red dwarf in its end cycle is quite difficult to find.
\newpage
\subsection{Yellow Dwarf}

\begin{figure}[h]
\centering
\includegraphics[width = 11cm]{sheet-YD}
\caption{Red giant's place in the classification}
\end{figure}


In terms of mass, the star just above Brown Dwarf is the Yellow Dwarf. Indeed its mass is between 0.08 and 1.2 mass of the sun. Its spectral type is G. The temperature at their surface is between 5000K and 6000K. These stars represent the previous stage of the Red Giant. They fuse the hydrogen into helium until there is no more hydrogen. When this happens, the Yellow Dwarf dies to become a Red Giant. The center of the star contracts as the external layers swell and cool to turn out red. 
The sun is a typical Yellow Dwarf.


\subsection{Red giant}
\begin{figure}[h]
\centering
\includegraphics[width = 10cm]{sheet-RG}
\caption{Red giant's place in the classification}
\end{figure}


Average stars, don't grow well, they expand and become the most voluminous stars of the universe. They have no more hydrogen to fuse in their core and are forced to ignite a reaction into the surroundings of core, or the outer core. This forces the star to expand quickly, up to 200 times bigger and 1000 times brighter than the Sun is to this day, but only up to 10 time its mass. They can even be up to three times lighter than the sun. 

\paragraph*{}
As far as star go, Red giants are rather cold because of their size. As told in the ideal gas status equation (page \pageref{GP}, we know that if the pressure $P$ decreases (just like a baloon, if the star expands, the pressure decreases, because we are talking here about the outer pressure, pressing \textit{on} the star to keep it as small as possible), the temperature will decrease.
The Sun will morph into a red giant in 5 billion years, it will turn to a huge star. This might destroy the current planetary system we live in, and will make all life on earth extinct\footnote{This shouldn't be at all pessimistic, because 5 billion years is more than earth itself. It is estimated that life took about 1 billion years to appear in the form of a small photosynthetic bacteria. It took us 3 billion years old to bacteria to what we are now, where will 5 more billion years lead us ?}

\subsection{Blue Giant}

\begin{figure}[h]
\centering
\includegraphics[width = 12cm]{sheet-BG}
\caption{Blue Giant's place in the classification}
\label{SheetBG}
\end{figure}
 
Some stars, a little more massive than those who become a red giant, become a blue giant. These stars are at least 10 times the size of the sun. They are highly brilliant and very hot. Also very massives, they consume their hydrogen very fast so that they lifetime is much lower: from 10 to 100 million years (remember that in general, the more massive, the shorter lifetime). They act like red giant but after fusionning hydrogen and helium she grows and cool down to become a Red Supergiant. These stars are the most massive ones compared to red giants, they will usually ultimately become supernovae and finally Black Holes. 
\paragraph*{}
The giant stars are basically the box where all the biggest stars are put, therefore, they have a huge amount of outcomes. As you can see in the classification, (figure \ref{SheetBG}) their mass goes from 2 to 150 solar masses. The gap between the lighest blue giants and the most massive is gigantic. The lighest ones will form a red giant, go back the the blue giant stage, then explode into a supernovae to finally form a black hole, while the heaviest ones will simply explode into a supernovae and form a nebula. It is believed that a quite massive blue giant that exploded into a supernovae is what originally created our solar system. This is why we can find so many rich and heavy elements in our planet, all were created in this supernovae explosion.

\subsection{White dwarf}

\begin{figure}[h]
\centering
\includegraphics[width = 12cm]{WD-sheet}
\caption{The place of the white dwarf in the classification }
\end{figure}

\label{whiteDwarf}
A white dwarf is the remains of non-massive star (e.g red giants) whose fusion has stopped: after merging hydrogen into helium and helium into carbon and oxygen, the gravity is not compensated by the energy of the fusion and the star collapses on itself. But to deceive death and avoid collapsing completely into a black hole, the star still has a trick up her sleeve. Electrons have a very strong repulsive force one towards another. The star then reaches another equilibrium point called a white dwarf.


\paragraph*{}
 It is now 1 million times denser than before\footnote{For example, if a mobile phone was as dense as a white dwarf it would weigh approximately 10 tons.}. Its density reaches 1 ton per $cm^2$. The White Dwarf usually have the size of planet Earth but their mass is approximately that of the sun.

\paragraph*{}

Since massive stars are very rare, this a the fate of the $96\%$ non massive stars of our galaxy. 
Nevertheless it keeps shining thanks to the energy stored by the star it once was. Indeed, the reduced surface of the star enables the heat transfer to be also reduced and slow, so that the star can keep the energy longer and delay its certain death.
\paragraph*{}

Then the star starts to decline slowly, its brightness diminishes and its gets colder and colder until the star dies. This process can take a very log time, even from the universe's point of view. 
The white dwarf are the embers of Red Giant: they are still shining but far less than their parent star but declining slowly.
This decline is extremely slow and ultimately, that the \textbf{white dwarf} will become a \textbf{black dwarf} within around 10 billions years. The universe is too young\footnote{The universe is approximately 13,7 billion years old.} to know such Black Dwarf. 


\begin{figure}[h]
\centering
\includegraphics[width = \textwidth]{time}
\caption{Time scale of a star's life cycle in the universe}
\end{figure}

If the star is a massive one, meaning at least 10 times the mass of the sun, its destiny is very different. They explode in supernovae\footnote{Explained later in this paper}. They also are very difficult to find because it is a rare phenomenon: around 2 or 3 supernovae per galaxy per century. To find one, scientist have to scan 1000 galaxies with a telescope that takes pictures every night of these galaxies and compare to the previous pictures too find some supernovae. The white dwarves are much more frequent.





\subsection{Variable Stars}

At the red giant stage or yellow giant stage some stars become variable stars. Though There are several types of variable stars, the majority of them are unstable star whose brightness varies because of volume variations. Indeed, stars are mainly very stable in their main sequence but when they get to the Red Giant stage, they become highly unstable for a while. 





\paragraph*{}
The balance between gravity and pressure is broken by a complex mechanism, so that sometimes the gravity wins for a while, the planet contracts itselfs until the pressure takes back the upper side and the star dilates. These volume variation provokes temperature fluctuations. When the star contracts the temperature rises and when the star dilates the temperature lowers. This inducts brightness variations. 
\paragraph*{}
One of the known variable star, Puppis\footnote{Observed by Hubble telescope}, was observed to change brightness by a factor of 6 within 40 days ! The gas expanding around those stars create an optical illusion called a light echo: it seems that the star is pulsing, but the variation of volume cannot be seen for afar: they are too small, only the brightness difference is measured.

\begin{figure}[h]
\centering
\includegraphics[width = 8cm]{variable-star}
\caption{Visualization of a light echo in a variable star}
\end{figure}

The period\footnote{Term very used in science to descrive periodical systems, meaning ``Time that an object takes to return in the exact same settings as it was''}  of these star goes from 1 day to 135 day. The interesting thing about the variable stars is that the brightness depends only on its position in the periodical cycle, and the position depends on the brightness. It is then possible to determine exactly the position of the star and so to draw a map of the universe based on the variable stars.
 
\paragraph*{} 
But variable stars aren't always periodical, for another type is an aperiodic one : the \textbf{eruptives} stars : as suggested by their name these stars undergo eruptions on their surface. The brightness varies only because of these \textit{eruptions} and they are very hard to predict. Since we experience lots of trouble trying to predict eruptions on earth, is seems impossible to predict one on another star, billions of billions of miles away.

\paragraph*{}
The last type of variable stars is the \textbf{novae}: some white dwarf in a binary system can become novae. If they are associated closely with a red giant in a binary star-system, the external layers of the red giant are attracted by the white dwarf. This phenomenon is called \textbf{matter transfer} and an accretion disk of spinning matter is formed around the white dwarf. This matter is composed mainly of hydrogen, which will ultimately be attracted by the white dwarf and will form a hot and dense layer around it. The brightness increases until the phenomenon stops. This type of variable star is not periodic and every change to the system is irreversible. 


\subsection{Supernovae}

\begin{figure}[h]
\centering
\includegraphics[height = 6cm]{supernova}
\caption{A 3D model of a supernovae}
\end{figure}

A supernovae is not \textit{per say} a class of star, this is why we will not give them a star classification place, but rather an event that occurs to a massive star, sometimes in somme classifications, unstead of putting supernovae in the classification, they put the star as ``potential supernovae star''.

\paragraph*{}

The supernovae are found approximately at 100 billions light year away from our solar system. They were really important for the beginning of earth life because all the heavy elements were formed by the supernovae. Indeed only a star with such power could produce the heavier elements. At the beginning scientist only knew that the first one (until iron) were produced by the red giants in the fusion process but were unable to determine where did the heavier elements (like lead, or uranium or even copper) were created.

\paragraph*{}

 But in 1957, in the article B$^2$FH\footnote{\textit{Synthesis of elements in stars}}, a team of scientists\footnote{Geoffrey Burbidge, Margaret Burbidge, Fred Hoyle and William Fowler} exposed for the first time the theory that every element on earth was produced by supernovae. In the following years, they managed to find proof of this by analyzing the spectrum of the light at the moment of a supernovae explosion captured by a telescope. They found more than 90 elements within the spectrum ! When the elements burnt they gave off different colors, for example potassium shades the flames in purple, strontium in red, sodium in yellow, copper in green/blue. By diffracting the spectrum it is possible to determine the elements that compose the star's vestigial cloud.

\paragraph*{}
When the star has enough mass, it is layered with different fusion process when you move closer to the core. Of course, all stars are like this but with heavier ones, the separation is more clear.

\begin{figure}[h]
\centering
\includegraphics[height = 6cm]{layers}
\caption{Different layers of a massive star in the end of its life fusion cycle}
\end{figure}

\paragraph*{}
The red giant\footnote{All massive stars, at the end of their fusion stages start growing and gets bigger and bigger, it is called a red giant.} at the end of the fusion sequence has an iron core, then a layer of helium, then a layer carbon, oxygen, neon,  magnesium, silicon, and finally neon. Each element produced is the fuel for the deeper layer's  fusion. This fusion process stops when the most stable element is created : \textbf{iron}. The star then has the mass of the earth and no longer has any energy source available to fight against gravitational forces. The core of the star contracts then relaxes and hits its outer layer. By a chain reaction, each layer hit its outer layer and finally the whole star explodes. The speed and the violence of the blast creates enough energy to create by fusion\footnote{Remember that fusion is actually created by atoms colliding, and the shock is creating new elements.} the other elements like zinc, gold, silver, argon, etc... The explosion spreads them in the whole cosmos, and it \textit{is} how all the elements heavier than iron were created on earth.

\subsection{After the supernovae : the neutron star}

\begin{figure}[h]
\centering
\includegraphics[width = 10cm]{sheet-NS}
\caption{Neutron star's place in the classification}
\end{figure}
The supernovae leaves behind it the remains of the star: a neutron star. The scientist predicted the existence of these stars theoretically before observing the first one. They observed in 1967 the first neutron star thanks to radio astronomy. By analyzing the different radio signal emitted in the universe, they discovered a precise periodic signal. The research led to the Crab nebula. A neutron star, also called pulsar is in the core of the residues of a supernovae.

\subsection{Pulsar}

\begin{figure}[h]
\centering
\includegraphics[height = 6cm]{crab-nebula}
\caption{The crab nebula pulsar}
\end{figure}

 The pulsar are very dense stars: a handful of matters would weigh as a small planet. They are so dense that the protons and the electrons in the stars merge to form neutrons\footnote{Remember that the white dwarf doesn't contract more because electrons exerce a repulsive force from one toward another. Neutron are much less repulsive.}. The pulsar weights between 1,4 time and 3 times the mass of the sun and is contained in a star whose diameter goes from 10 km to 20 km ! Because of the collapsing of the previous star, the rotation speed and the magnetic field also reaches incredible values ! 
After the supernovae explosion, the star becomes very small and by the conservation of the kinetic moment the speed is increasing. It's just like a ice skater folding his arms: he turns faster\footnote{Imagine then the difference between a star bigger than the sun shrinking to 10km in diameter, the increase in rotation speed is huge.}. Moreover because of the conservation of the magnetic flux if the surface is reduced and the magnetic field increases. Pulsars are big magnets turning really fast.

A normal star has a magnetic field of about 1 gauss, a pulsar has a magnetic field of some \textbf{thousands gauss}. Ergo, it emits a regular magnetic pulse that can be measured even from earth's outer atmosphere. Their regularity is extremely precise, which make scientist think that they are the best clock, even better than atomic clocks. In the figure 7, the blue lines are the magnetic field and the blue halo is the radiations emitted by the pulsar. On earth, it is this halo that is measured periodically. To spin around faster some pulsar associate with other with another star: for example with a red giant. The matters outflow from a star can reach the spinning matter around the pulsar and hit the pulsar. This gives more and more impulsion to the pulsar: in this case, the pulsar is called a millisecond pulsar. We even discovered a a binary star system of a planet with a pulsar. You can see a 3D modeling of such a system in figure \ref{bin}. There is another thing than can affect the speed of a pulsar: the crust of a neutron star is extremely resistant, but the inside of the star is only gas. Just like earth's earthquaques, forces inside the star can make the crust crack, and cause a \textit{starsquake}. When it cracks, the neutron star readjusts and changes its rotation\footnote{this phenomenon is spontaneous, to maintain a balanced state of the pulsar}. This phenomenon is called a glitch. 




                          
\begin{figure}[h]
\centering
\includegraphics[height = 6cm]{binary-system}
\caption{Binary star system with a pulsar and a flow of matter in the center of mass}
\label{bin}
\end{figure}
                            
But even within the pulsars, there are different types. A specific kind of pulsar has been recently discovered and only 15 have been found over 400 billions stars in our galaxy, which makes this type of star extremely rare. They are called \textbf{Magnetars}. They are similar to the neutron stars with a magnetic field pushed to the extreme. 1000 millions of millions time higher than the one on Earth. If such a magnetic field was even in a solar system nearby ours, it wouldn't allow life to be, because it tends to align the water molecules as show in the figure \ref{water}. The magnetars remain a mystery for the scientists, because it is a very rare phenomenon to observe, and to gather data on.



\begin{figure}[h]
\centering
\includegraphics[height = 5cm]{water}
\caption{water molecules aligned because of a high magnetic field}
\label{water}
\end{figure}

If the supernova turn into a neutron star whose mass is more than 3 times the mass of the sun, they are called Black Holes. These star are so dense that even light cannot escape from the gravitational force. 



\subsection{Black Holes}
A black hole is a very dense neutron star. Two kinds of Black Holes exist: the active ones and the non-active ones. An active Black Hole is powered by matters that spins around it before the hole sucking all of it. While spinning, the gases rub against each other and heats until millions of degrees. That is why they emit light just before they disappear in the black hole. Next to these brilliant gases, the ``horizon limit'' is the line between the light and the darkness of the hole. Beyond this limit nothing passes: no matter, no light, no information, nothing we could see or measure. Often, the active black hole present a privileged direction of light as shown in the following picture.
\newpage
\begin{figure}[h]
\centering
\includegraphics[height = 4cm]{active-black-hole}
\caption{Active black hole}
\end{figure}


This stellar object is as the pulsar, submitted to a very high magnetic field. The charged particles in the hot gases around the black hole are attracted by this magnetic field and follow it. It creates an outflow of matters that emit light perpendicularly to the disc of matter. For a long time scientist thought there were 3 types of quasar: the dragens, the blazars and the quasar. However the suppositions are now that they are all the same object seen in different angles.

When an active black hole has no matter left around it, it starts spinning off and remains dark. Research on black holes are crucials to the understanding of the universe: the pre Big Bang singularity was also an extremely dense point that can be thought as an extremely small and tremendously dense star. Understanding one may very well lead to the understanding of the other. 


\section*{Conclusion}

The life of a star is mesmerizing, and there are lots of different types of star, lots of different criteria for stellar classification. The discovery and understanding of the basic process of the star allowed us to make tremendous advancements in the fields of science, and in energy production. Without stars, we wouldn't have undersstood quantum physics and the energy in small scales. Einstein often was thinking about stars and the cosmos after making discoveries, to test their validity. The cosmos is so rich that if he was not able to find evidence of what he was testing, he would think it did not even exist. 
\paragraph*{}
To take a step back of this classification, we would like the reader to notice that stars are really complicated, and we tried to put them in boxes from the beginning. However, for an example, if someone was looking at humans from a very distant point of view, the process is also simple, we breath to use oxygen for our muscles, and we eat to break some molecular bonds and gather energy. The only difference between stars and us is that stars look like a primitive lifeform, To this day, humans have evolved very far from were the first lifeform got, we had to evolve because we were threatened by other lifeforms, like predators. Star have no predator but themselves. They do not need to evolve and they can't pass anything to the stars that will be reborn in their ashes. To us, this raises a question : from a physical standpoint, apart from every philosophical point of view, where does life begin ? If you zoom in on us, it is just atoms moving, granted there are complicated reactions, but stars have tremendously complicated reactions too, why can we say that we are alive and they are not ? What is the rim between life and lifeless ? A physicist anylises what he sees, but will never be able to say this. Perhaps stars aren't so far from us as we think.

\newpage



\section*{Annex 1 : All the stages of a star's life}

To recap all being said in this paper, we made a little understanding visual graph of all the possibilities in a star's life-cycle, in function of mass.



\begin{figure}[h]
\centering
%\begin{tikzpicture}
%\node[inner sep=0pt] (Born) at (0,0) {\includegraphics[height=2cm]{early-star}} ; \draw (Born.south) node [below] {Birth of a star};
%\node[inner sep=0pt] (Neb) at (5,-5) {\includegraphics[height=2cm]{nebula}} ; \draw (Neb.south) node [below]{Nebula};
%\node[inner sep=0pt] (RG) at (-5,-5) {\includegraphics[height=2cm]{red-giant}} ; \draw (RG.south) node [below]{Red Giant};
%\node[inner sep=0pt] (Super) at (0,-10) {\includegraphics[height=2cm]{supernova}} ; \draw (Super.south) node [below]{Supernovae};
%\node[inner sep=0pt] (WD) at (-7,-9) {\includegraphics[height=2cm]{white-dwarf}} ; \draw (WD.south) node [below]{White dwarf};
%\node[inner sep=0pt] (NS) at (6.7,-10) {\includegraphics[height=2cm]{neutron-star}} ; \draw (NS.south) node [below]{Neutron star};
%\node[inner sep=0pt] (BH) at (5.5,-14) {\includegraphics[height=2cm]{black-hole}} ; \draw (BH.south) node [below]{black hole};

%\draw [->,very thick] (Born.west)[pos=0.1] to[bend right] (RG.north) [pos=0.8] ;
%\draw [<-,very thick] (Born.east) to[bend left] (Neb.north) ;
%\draw [->,very thick] (RG.west) to[bend right] (WD.north) ;
%\draw [->,very thick] (RG.west) to[bend right] (WD.north) ;
%\draw [->,very thick] (RG.320) to[bend right] (Super.west) ;
%\draw [<-,very thick] (Neb.220) to[bend left] (Super.east) ;
%\draw [->,very thick] (Super.east) to[bend right] (NS.west) ;
%\draw [->,very thick] (NS.220) to[bend right] (BH.north) ;

%\end{tikzpicture}
\includegraphics[trim = {0 0 1mm 0},clip, width = \textwidth]{stages}
\end{figure}
\newpage
\section*{Annex 2 : Star classification in function of light}

There are several ways to make a star classification, all this document, we have explained that the key factor is mass, but mass cannot be measured from afar : you cannot weight an object from billions of kilometers away. The criteria used ten is the luminosity and color of the star, basically red stars are less massive stars, then yellow stars, then blue stars. All of this ei explained in the following figure	 :

\begin{figure}[h]
\centering
\includegraphics[width=\textwidth]{classification}
\caption{Star classification by light emission}
\end{figure}

\newpage

\section*{Annex 3 : Eye candy}
Just for the pleasure of the eyes, here are pictures of nebulas taken by the hubble telescope.

\begin{figure*}[h!]
    \centering
  \begin{subfigure}[b]{0.45\textwidth}
      \includegraphics[width=6cm]{1}
  \end{subfigure}%
  ~ 
  \begin{subfigure}[b]{0.45\textwidth}
      \centering
      \includegraphics[width=6cm]{2}
  \end{subfigure}
 ~
  \begin{subfigure}[b]{0.45\textwidth}
      \centering
      \includegraphics[width=6cm]{3}
  \end{subfigure}
  ~ 
  \begin{subfigure}[b]{0.45\textwidth}
      \centering
      \includegraphics[width=6cm]{4}
  \end{subfigure}
  ~
  \begin{subfigure}[b]{0.45\textwidth}
      \centering
      \includegraphics[width=6cm]{5}
  \end{subfigure}
  ~ 
  \begin{subfigure}[b]{0.45\textwidth}
      \centering
      \includegraphics[width=6cm]{6}
  \end{subfigure}
  
\end{figure*}

\newpage

\subsection*{References}
Articles :
\begin{itemize}
\item \href{http://www.schoolsobservatory.org.uk/learn/astro/stars/cycle}{School Observatory - Stars cycles}.
\item \href{https://futurism.com/the-life-cycle-of-a-star/}{Futurism - The life cycle of a star}.
\item \href{http://expositions.bnf.fr/ciel/elf/page12.htm}{National France Library - Understanding the star's life}
\item \href{http://www.cnrs.fr/cw/dossiers/dosbig/decouv/xchrono/etoiles/niv1_1.htm}{France National Research Center - Star theory}
\end{itemize}

Videos :
\begin{itemize}
\item Arte videos - the life of a star, the star's fuel.
\item Discovery channel - The Universe
\end{itemize}

%\bibliographystyle{unsrt}

%\bibliography{sample}

%----------------------------------------------------------------------------------------

\end{document}